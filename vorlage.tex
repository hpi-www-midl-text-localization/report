% Koma-Script Basisklasse
\documentclass[a4paper,12pt,pagesize,headsepline,bibtotoc,titlepage]{scrartcl}

\usepackage[ngerman]{babel}		% deutsche Trennmuster
\usepackage[utf8]{inputenc}		% direkte Eingabe von Umlauten & Co. (Vorsicht: Encoding im Editor muss auch UTF-8 sein!)

\usepackage[T1]{fontenc}			% T1-Schriften

\usepackage{mathptmx}			% Times/Mathe \rmdefault
\usepackage[scaled=.90]{helvet}	% Skalierte Helvetica \sfdefault
\usepackage{courier}			% Courier \ttdefault

% Zusatzpakete für mehr mathematische Symbole, Einfügen von Grafiken 
% und bessere Bildunterschriften
\usepackage{amsmath,amsthm,amsfonts,graphicx,caption}

% Wenn man direkt mit dem pdflatex eine PDF-Datei erzeugt, sollten diese beiden Pakete eingebunden werden
\usepackage{hyperref} % Hyperlinks anklickbar
\usepackage{ae,aecompl} % bessere Bildschirmschriftarten usw.
\usepackage{epstopdf} % support eps 

\pagestyle{headings}

% Abstand der Kopfzeile vom Text:
\headsep4mm

\typearea[current]{current}     % Satzspiegel neu berechnen

% andere Bildunterschrift mit Hilfe von caption
\renewcommand{\figurename}{Abb.}
\renewcommand{\captionlabelfont}{\bf}

\title{
	\includegraphics*[width=0.4\textwidth]{hpi_logo_2017.eps}\\
	\vspace{24pt}
	Titel des Themas
}
\subtitle{
	Lecture\\
	Machine Intelligence with Deep Learning\\
	Wintersemester 2018/2019
}
\author{
	Dein Name hier\\[12pt]
	Betreuer:\\
	Betreuer Name,\\	
	Dr. Haojin Yang
}
\date{\today}

\begin{document}
\maketitle
\tableofcontents
\newpage


\section{Kapitelüberschrift}
Lorem ipsum dolor sit amet, consectetur adipiscing elit. Morbi sed nunc leo. Nam ac leo venenatis est commodo vehicula. Nulla justo nisl, venenatis id tincidunt id, porta ut felis. Cras eu justo ac nisi ornare commodo placerat at risus. Suspendisse ut urna tellus. Cras ut erat tempus justo aliquam laoreet. Praesent est neque, interdum quis convallis et, gravida sed arcu. Mauris bibendum, dui at ullamcorper luctus, arcu dolor laoreet nisi, ut facilisis dui enim a nisl. Sed odio risus, pulvinar suscipit feugiat pharetra, varius non est. Morbi pellentesque libero eu odio pulvinar semper. Praesent cursus adipiscing metus nec fermentum. Nullam malesuada euismod mi nec tincidunt. Nulla eget auctor velit. Mauris quam odio, blandit sit amet pharetra non, lobortis sed risus. Aliquam nec orci vel dolor suscipit tristique. Etiam quam eros, commodo at iaculis at, molestie eu metus. Maecenas viverra dui non magna suscipit sodales commodo justo iaculis (siehe Abbildung \ref{abb:test}).

\begin{figure}[hbp]
\begin{center}
\includegraphics*[width=0.75\textwidth]{beispiel.png}\\
\caption{Eine Abbildung, Quelle: \cite{BMXNet17}}
\label{abb:test}
\end{center}
\end{figure}

\subsection{Abschnittsüberschrift}
Quisque pharetra, tellus id cursus consectetur, libero arcu sollicitudin nisl, in pulvinar tortor tortor consequat arcu. Donec pulvinar nibh in elit iaculis vel pellentesque diam tincidunt. Fusce posuere volutpat libero in consequat. Curabitur sit amet interdum leo. Vestibulum massa ante, ultrices vel dictum vitae, vestibulum consequat magna. Pellentesque turpis ligula, fermentum vel consectetur id, ornare non diam. Lorem ipsum dolor sit amet, consectetur adipiscing elit. Vestibulum pretium scelerisque nisi sit amet semper. Nulla lobortis quam sit amet turpis tristique accumsan. Vivamus sed nunc odio.

In tempor leo ut lectus lacinia commodo. Vestibulum pretium accumsan erat, non interdum sem feugiat a. Praesent tempor odio quis arcu laoreet dignissim. Mauris egestas lectus sed purus malesuada feugiat. Lorem ipsum dolor sit amet, consectetur adipiscing elit. Donec placerat auctor laoreet. Nunc hendrerit viverra posuere. Morbi egestas ante et augue mattis eleifend. Aliquam nibh turpis, adipiscing ac lacinia sed, laoreet vel orci. Donec velit est, fringilla ac vulputate vitae, commodo a velit. Fusce vitae posuere diam. Pellentesque tellus tellus, volutpat non fringilla in, vulputate vitae magna. Proin ligula diam, aliquet eget feugiat ac, rutrum in tortor.



\newpage
%%%% bibtex file %%%%%%
\bibliographystyle{alphadin}
\bibliography{references} 

\end{document}